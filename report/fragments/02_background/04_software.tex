\section{Related Software and Tools}

\subsection{C++ Standard Template Library}
The C++ Standard Template Library (C++ STL from now on) is a library for the C++ language which provides the developer with a set of frequent use classes, divided into four categories: \textit{algorithms}, \textit{containers}, \textit{functions}, and \textit{iterators}. These classes are made available through the use of C++ templates, which allow compile-time polymorphism, which is by definition more efficient and lightweight than run-time polymorphism. This library has the benefit that most compilers have already optimized routines to accelerate C++ STL based code during the compilation stage.

\subsection{C++ Standard Library}
The C++ Standard Library is a collection of classes, structures, and functions which specifies the semantics of generic algorithms using C++ ISO standard heavily influenced by C++ STL. This library is made available as a set of headers which the developer can include into their code to make use of already optimized routines available as binaries by the underlying operating system. It also provides an abstraction layer to invoke C libraries, and it is part of the \textit{C++ ISO Standardisation} effort.

\subsection{Boost.org}
Boost Library is a free, open source and platform-wide available set of general purpose operations implemented on C++ for many application fields. Boost has 167 libraries up to the date (year 2020) which range from smart pointer structures, image processing operations, advanced threading management, and so on. From those libraries, ten of them are already included on the Library Technical Report (TR1) for continuous standardization. It is developed by Boost.org, which is an community-led organization supports research and education into the best possible uses of C++ and libraries.

\subsection{SciPy environment}
SciPy is a Python-based ecosystem of open-source software for mathematics, science, and engineering. With scientific computing as their primary field of application, it includes libraries to support the use of data frames via \textit{pandas}, statistics operation and formulas via \textit{scipy}, interactive notebook environments like \textit{Jupyter} which uses \textit{IPython} as its core, accelerated math operations with \textit{numpy}, among many other tasks. 

We use \textit{Jupyter} as out primary tool for organizing our experiments, which allows usage of IPython commands via a browser or other IDEs using instances called \textit{kernels}. Each kernel gets allocated their resources dynamically and it is not restricted to execute python code, as it can execute external commands, shell instructions, among other dialects. 

\subsection{Valgrind}
Valgrind is an instrumentation framework for building dynamic analysis tools, popular with C developers due to its direct integration with LLVM pipeline. Valgrind is typically used as a debugging tool to diagnose and detect memory problems such as memory leaks, stray associations, threading bugs, and program profiling.